\documentclass[oupdraft]{bio}
% \usepackage[colorlinks=true, urlcolor=citecolor, linkcolor=citecolor, citecolor=citecolor]{hyperref}

\usepackage{subfigure, enumitem}
\usepackage{url}

% Add history information for the article if required
\history{Received August 1, 2010;
	revised October 1, 2010;
	accepted for publication November 1, 2010}

\begin{document}
	
	% Title of paper
	\title{A tutorial on the ``longpower" R package and a web-based App for conducting power analysis and sample size computation for longitudinal data}
	
	% List of authors, with corresponding author marked by asterisk
	\author{SAMUEL IDDI$^1$, MICHAEL C. DONOHUE$^{1\ast}$\\[4pt]
		for the Alzheimer's Disease Neuroimaging Initiative$^\dagger$\\[4pt]
		% Author addresses
		\textit{$^1$ Data Measurement and Evaluation Unit,
			African Population and Health Research Center (APHRC),
			Nairobi, Kenya\\
		$^2$ 	Alzheimer's Therapeutic Research Institute,
		Keck School of Medicine,
		University of Southern California,
		San Diego, USA}
		\\[2pt]
		% E-mail address for correspondence
		{siddi@aphrc.org}}
	
	% Running headers of paper:
	\markboth%
	% First field is the short list of authors
	{S. Iddi and others}
	% Second field is the short title of the paper
	{A tutorial on the ``longpower" package and web-based App}
	
	\maketitle
	
	% Add a footnote for the corresponding author if one has been
	% identified in the author list
	\footnotetext{	$^\dagger$ Data used in preparation of this article were obtained from the Alzheimer's Disease Neuroimaging Initiative (ADNI) database (adni.loni.usc.edu). As such, the investigators within the ADNI contributed to the design and implementation of ADNI and/or provided data but did not participate in analysis or writing of this report. A complete listing of ADNI investigators can be found at: \url{http://adni.loni.usc.edu/wp-content/uploads/how_to_apply/ADNI_Acknowledgement_List.pdf}}
	
	\begin{abstract}
		{Longitudinal randomized trials are common in medical and clinical research. These trials often require the collection of data on the same individual at different data collection waves. At the design phase of such studies, sample size computations are critical to ensure that studies are sufficiently powered to provide reliable and valid inference. There are several methodologies for calculating sample sizes for longitudinal studies that depend on many considerations including the type of research design, outcome type, and proposed analytical methods. More advanced procedures take into account several factors leading to more complicated formulas for computing sample sizes. To provide easy access to common and widely used sample size and power calculation formulas, this tutorial briefly describes some methods for longitudinal data.  We also enrich the discussion with real-life examples comparing treatment versus control groups in randomized trials assessing treatment effect on clinical outcomes. Accompanying this tutorial is also a web-based sample size shinyApp developed to help researchers to conduct different sample size and power calculations by allowing user-specified parameters or pilot parameters generated from the Alzheimer’s Disease Neuroimaging Initiative (ADNI) study.}
		{Mixed model for repeated measures, linear mixed model, shinyApp, power, sample size, longpower, longitudinal data}
	\end{abstract}
	
	
\section{Introduction}
\label{sec1}
Longitudinal designs are generally preferred over cross-sectional research design as they yield higher statistical power. As such, many biomedical and medical studies employ longitudinal design to study changes over time in outcome at the individual, group, or population level. In the early stages in the design of a longitudinal experimental or population study, it is imperative to ensure that sufficient participants are recruited to detect a predetermined effect size.  Inadequate sample sizes lead to invalid or inconclusive inference and a complete waste of resources (\cite{Lu_Methrotra_Liu(2009),Yan_Su(2006)}). On the other hand,  oversampling causing a waste of resources and exposure to many participants to harm that may be associated with the research (\cite{Lu_Methrotra_Liu(2009)}). Thus, optimal sample size and power analysis have become an important prerequisite for any quantitative research design. Not only are these required during the design phase of research, but it has also become mandatory when preparing protocols for ethical review and research grant applications to guarantee both economic and ethical benefits. 
Determining the right sample size for a study is not a straightforward task. Despite the plethora of sample size formulas for repeated measures (\cite{Overall_Doyle(1994),Lui(1992),Rochon(1991), Guo_etal(2013)}), cluster repeated measures (\cite{Liu_Shih_Gehan(2002)}), multivariate repeated measures (\cite{Vonesh_Schork(1986),Guo_Johnson(1996)}), longitudinal research designs (\cite{Lefante(1990)}), the tasks of gathering the necessary inputs and getting the right software to carry out the computation are challenging to many. Investment in such efforts is not what many non-technical researchers desire to embark on. Commonly, the easiest route taken by most research is to use sample size formulas for very basic cross-sectional studies and adjust for design effect due to repeated measures. Although such approaches are valid, the best approach is the use of formulas derived directly from models for longitudinal or repeated measures to align with planned data analysis and yielding greater statistical power. 
Several factors need to be considered before choosing the right sample size formulas that increase the statistical power of a study. \cite{Guo_etal(2013)} describe practical methods for the selection of appropriate sample size for repeated measures addressing issues of missing data, and the inclusion of more than one covariates to control for differences in response at baseline. 
	
Sample size formulas are refined depending on the specific situation and design features. For example, \cite{Hedeker_Gibbons_ Waternaux (1999)} considered a sample size for longitudinal design comparing two groups that accounted for subject attrition or drop-out. \cite{Basagana_Liao_Spiegelman(2013)} derived sample size formulas for continuous longitudinal data with time-varying exposure variables typical of observational studies. Ignoring time-varying exposure was demonstrated to lead to substantial overestimation of the minimum required sample size which can be economically disadvantageous. In non-traditional longitudinal designs such as designs for mediation analysis of the longitudinal study, further refinements to sample size formulas are needed to ensure that sufficient sample sizes are obtained for conducting mediation analysis (\cite{Pan_etal(2018)}). 
	
Extension to basic sample size formulae usually requires additional parameters such as exposure mean, variance, and intraclass correlations (\cite{Basagana_Liao_ Spiegelman(2013)}), mediation effect, number of repeated measures  (\cite{Pan_etal(2018)}), covariance structures (\cite{Rochon(1991) }), non-linear trends (\cite{Yan_Su(2006)}), missing, attrition or dropout rates (\cite{Roy_etal(2007), Lu_Luo_Chen(2008)}), among others.  
	
Advanced sample size methods simultaneously handle several practical issues associated with research design and complications that may arise during data collection. However, such methods are only available in commercial software. 



Sample size calculation and power analysis are important components in designing a new trial or study. For many non-statisticians, the processes involved in performing these types of analyses can be daunting. A major hurdle to overcome is the availability of pilot parameters which are required inputs for generating sample size and power outputs. In this paper, we develop a power/sample size App for Alzheimer's Disease (AD) clinical trials. The web-based app  implements the linear mixed model and basic mixed method repeated measures (MMRM) allowing user to input own pilot estimates, or use an ADNI-based pilot estimate generator to compute sample size and power. 


\section{Methods}
\label{sec2}
Clinical trial data collected longitudinally over time are commonly analyzed using the linear mixed model and mixed model for repeated measures methods for continuous outcomes. Prior to such trials, sample size and power analysis are conducted to obtain the required sample size needed to assess treatment effect with optimal power. Various sample size approaches for longitudinal data have been proposed. We review a few of the most commonly used methods applied in Alzheimer's disease trials focusing on continuous outcome type.

\subsection{Sample size computation based on the linear mixed model (LMM)}
\label{model}
One of the most common sample size computation approaches for correlated data is derived by \cite{Liu_Liang(1997)}. This approach derived sample size from a generalized estimating equation (\cite{Liang_Zeger(1986)}). Thus, different outcomes types can be handled. A special case is for continuous response measured repeatedly over time and modeled using a linear mixed model (LMM). The model specification involves fixed covariates and random-effects to the outcome for between-subject variability. The error component is assumed to follow a multivariate with a mean vector of zeros and covariance matrix. Different covariance structures such as independence, exchangeable, auto-regressive, and unstructured to estimate the minimum sample size for a given significant level and pre-specified nominal power. A reduced form of the linear mixed model with only the treatment effect and a random-intercept component, the sample size formula generated by this approach is equivalent to that of \cite{Diggle_Liang_Zeger(1994)}. 

\subsection{Sample size computation based on the mixed model for repeated measures (MMRM)}
\label{model2}
An alternative sample size computation method used in many areas of application including AD trials is that based on the mixed model for repeated measures (MMRM \cite{Mallinckrodt_etal(2001),Mallinckrodt_etal(2003),Lane(2008)}. The MMRM method of model fitting is similar to the mixed-effect model for longitudinal or repeated measures except for the unstructured modelling of time – treated as a categorical variable, and the specification of a within-subject error structure. Additionally, the MMRM considers the interaction between time and treatment. 

\subsection{The 'longpower' package}
\label{model3}

\subsection{A Web-based App}
\label{model4}
This ShinyApp dashboard is developed to easily generate sample size and conduct power analysis for a longitudinal study design with two-group comparisons for a continuous outcome. The App implements the sample size formula of \cite{Liu_Liang(1997)} and \cite{Diggle_etal(1994)} using functions developed for the R `longpower' package. The `longpower' package handles cases where time is treated either as continuous or categorical. The former approach uses the linear mixed model with random intercept and slope while the later leads to the well-known Mixed Model of Repeated Measures (MMRM) used in many clinical trial applications for conducting statistical analysis. 

The dashboard is in two parts. The first part accepts user inputs to generate sample size when time is treated as both categorical and continuous. Thus, this part assumes that the user already has pilot parameter estimates including effect size and known variance. Users can generate sample sizes and perform power analysis using different sample size methods (diggle, liuliang, and edland). The basic inputs of the App include an option for whether sample size or power is desired, a slider for sample size and power enabled when appropriate and difference in means (treatment effect) estimate. Additional options include the type I error rate, type of test (one/two sided), estimates of random effect variances, sample size computation method, allocation ratio and time intervals. For the MMRM method, options for the association structure and retention in each group is enable depending on the association structure assumed by the user. 

The second part of the dashboard uses the methodology that is similar to the first part except that Alzheimer's Disease Neuroimaging Initiative (ADNI)-based pilot parameters are generated based on user-selected inclusion and exclusion criteria, primary outcome, duration of the study, and covariate options to be included in the linear mixed and MMRM models. 

The App interface display plots of power against sample size over range of values, and text summary of imputed and selected estimates. For the ADNI-based generator, additional summary estimates from fitted model and descriptive summary of baseline participants characteristics are also dynamically generated.


\section{Motivating examples}
\label{sec3}

\section{Discussion}
\label{sec4}




\section*{Acknowledgments}

We are grateful to the ADNI study volunteers and their families. This work was supported by Biomarkers Across Neurodegenerative Disease (BAND-14-338179) grant from the Alzheimer's Association, Michael J. Fox Foundation, and Weston Brain Institute; and National Institute on Aging grant R01-AG049750. Data collection and sharing for this project was funded by the Alzheimer's Disease Neuroimaging Initiative (ADNI) (National Institutes of Health Grant U01 AG024904) and DOD ADNI (Department of Defense award number W81XWH-12-2-0012). ADNI is funded by the National Institute on Aging, the National Institute of Biomedical Imaging and Bioengineering, and through generous contributions from the following: AbbVie, Alzheimer's Association; Alzheimer's Drug Discovery Foundation; Araclon Biotech; BioClinica, Inc.; Biogen; Bristol-Myers Squibb Company; CereSpir, Inc.; Eisai Inc.; Elan Pharmaceuticals, Inc.; Eli Lilly and Company; EuroImmun; F. Hoffmann-La Roche Ltd and its affiliated company Genentech, Inc.; Fujirebio; GE Healthcare; IXICO Ltd.; Janssen Alzheimer Immunotherapy Research \& Development, LLC.; Johnson \& Johnson Pharmaceutical Research \& Development LLC.; Lumosity; Lundbeck; Merck \& Co., Inc.; Meso Scale Diagnostics, LLC.; NeuroRx Research; Neurotrack Technologies; Novartis Pharmaceuticals Corporation; Pfizer Inc.; Piramal Imaging; Servier; Takeda Pharmaceutical Company; and Transition Therapeutics. The Canadian Institutes of Health Research is providing funds to support ADNI clinical sites in Canada. Private sector contributions are facilitated by the Foundation for the National Institutes of Health (www.fnih.org). The grantee organization is the Northern California Institute for Research and Education, and the study is coordinated by the Alzheimer's Therapeutic Research Institute at the University of Southern California. ADNI data are disseminated by the Laboratory for Neuro Imaging at the University of Southern California.
{\it Conflict of Interest}: None declared.


\bibliographystyle{biorefs}
\bibliography{refs}

	
	

\end{document}